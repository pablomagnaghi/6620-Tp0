\documentclass[a4paper,10pt]{article}

\usepackage{graphicx}
\usepackage[ansinew]{inputenc}
\usepackage[spanish]{babel}
\usepackage{listings}

\title{		\textbf{Trabajo Pr\'{a}ctico 0: Infraestructura b\'{a}sica}}

\author{	Hern\'{a}n Gonz\'{a}lez, \textit{Padr\'{o}n Nro. 79.460}		\\
            \texttt{ gonzalezhg@yahoo.com.ar }						\\[2.5ex]
            Pablo Magnaghi, \textit{Padr\'{o}n Nro. 88.126}				\\
            \texttt{ pablomagnaghi@gmail.com }						\\[2.5ex]
            Enzo Guagnini, \textit{Padr\'{o}n Nro. 88.325}				\\
            \texttt{ enzog\_m@hotmail.com }						\\[2.5ex]
            \normalsize{1er. Cuatrimestre de 2011}					\\
            \normalsize{66.20 Organizaci\'{o}n de Computadoras  $-$ Pr\'{a}ctica Martes}\\
            \normalsize{Facultad de Ingenier\'{i}a, Universidad de Buenos Aires}	\\
       }

\date{19/04/2011}

\begin{document}

\maketitle

\thispagestyle{empty}

\newpage
\newpage

\section{Introducci\'{o}n}
Este art\'{i}culo se refiere a la implementaci\'{o}n de un programa desarrollado en lenguaje C, el cual es una versi\'{o}n simplificada del comando \texttt{tr} de UNIX.

El mismo copi\'{a} caracteres desde \textit{stdin} a \textit{stdout} realizando sustituciones o eliminando caracteres seleccionados a trav\'{e}s de los argumentos del programa, detectando e imprimiendo errores por \textit{stderr} en caso de haberlos.

\section{Enunciado}

\subsection{Objetivos}
Familiarizarse con las herramientas de software que usaremos en los siguientes  trabajos, implementado un programa (y su correspondiente documentaci\'{o}n) que resuelva el problema piloto que se presentar\'{a} m\'{a}s abajo.
	
\subsection{Alcance}
Este trabajo pr\'{a}ctico es de elaboraci\'{o}n grupal, evaluaci\'{o}n individual, y de car\'{a}cter obligatorio para todos alumnos del curso.

\subsection{Requisitos}
El trabajo deber\'{a} ser entregado personalmente, en la fecha estipulada, con una  car\'{a}tula que contenga los datos completos de todos los integrantes. Adem\'{a}s, es necesario que el trabajo pr\'{a}ctico incluya (entre otras cosas), la presentaci\'{o}n de los resultados obtenidos, explicando, cuando corresponda, con fundamentos reales, las causas o razones de cada resultado obtenido.

El informe deber\'{a} respetar el modelo de referencia que se encuentra en el grupo, y se valorar\'{a}n aquellos escritos usando la herramienta TEX / LATEX.

\subsection{Recursos}
Usaremos el programa GXemul para simular el entorno de desarrollo que utilizaremos en este y otros trabajos pr\'{a}cticos, una m\'{a}quina MIPS corriendo una versi\'{o}n reciente del sistema operativo NetBSD. 

\newpage
\newpage

\subsection{Implementaci\'{o}n}
\subsubsection{Programa}
El programa a escribir en lenguaje C, es una versi\'{o}n minimalista del comando \texttt{tail} de UNIX. El mismo copia caracteres desde \textit{stdin} a \textit{stdout} realizando sustituciones o eliminando caracteres seleccionados a trav\'{e}s de los argumentos del programa.
La funcionalidad a desarrollar consiste en:

\begin{itemize}
\item Traducci\'{o}n de caracteres.
\item Sumarizaci\'{o}n de caracteres.
\item Borrado de caracteres.
\end{itemize}

El programa recibe dos argumentos, \textit{string1} y \textit{string2}, y los usa para operar sobre los

El programa recibe dos argumentos, \textit{string1} y \textit{string2}, y los usa para operar sobre los caracteres leidos de \textit{stdin}.La traducci\'{o}n de caracteres se realiza a partir de la posici\'{o}n de un caracter leido de \textit{stdin} en \textit{string1}, reemplazandolo por el caracter en la misma posici\'{o}n de \textit{string2}.
La sumarizaci\'{o}n de caracteres se realiza a partir de los caracteres contenidos en \textit{string1}. Si un caracter leido de \textit{stdin} est\'{a} contenido en \textit{string1}, entonces el mismo debe escribirse en \textit{stdout} una sola vez.
Para el borrado de caracteres no se escriben en \textit{stdout} aquellos que son leidos de \textit{stdin} y est\'{a}n contenidos en \textit{string1}.
El programa devuelve 0 en caso de exito. Todos los errores deben escribirse en \textit{stderr}.

\subsubsection{Consideraciones}

\subsubsection{Combinaci\'{o}n de tareas}
Estas tareas pueden o no estar superpuestas de la siguiente manera:\\
\indent\texttt{tp0 -s string1 string2}: La sumarizaci\'{o}n de caracteres ocurre luego de la traducci\'{o}n.\\
\indent\texttt{tp0 -sd string1 string2}: Se eliminan los caracteres contenidos en \textit{string1} y se sumarizan los contenidos en \textit{string2}.

\subsubsection{Sobre la longitud de los argumentos}
En caso de que la longitud de \textit{string1} sea mayor a la longitud de \textit{string2}, entonces se debe extender \textit{string2} con su \'{u}ltimo caracter hasta igualar las longitudes.

\subsubsection{Ocurrencias en el primer argumento}
En caso de que un caracter se repita en \textit{string1}, debe tomarse la \'{u}ltima ocurrencia al realizar la traducci\'{o}n.

\newpage
\newpage

\subsubsection{Ejemplos}
Primero, usamos la opci\'{o}n \indent\texttt{-h} para ver el mensaje de ayuda:

\indent\texttt{\$ tp0 -h}\\
\indent\texttt{Usage:}\\
\indent\texttt{tp0 -h}\\
\indent\texttt{tp0 -V}\\
\indent\texttt{tp0 [options] string1 string2}\\
\indent\texttt{tp0 [options] string1}\\
\indent\texttt{Options:}\\
\indent\texttt{-V, --version Print version and quit.}\\
\indent\texttt{-h, --help Print this information and quit.}\\
\indent\texttt{-d, --delete Delete characters in string1}\\
\indent\texttt{-s, --squeeze Squeeze characters in input.}\\

\indent\texttt{Examples:}\\

\indent\texttt{tp0 abc 123 < alpha.txt}\\
\indent\texttt{tp0 -ds 123456789 aeiou}\\

\indent\texttt{\$ echo ``3.14159192'' | tp0 1 9}\\
\indent\texttt{3.94959992}\\

\indent\texttt{\$ tp0 aba 123 < text.txt}\\
\indent\texttt{323}\\

\indent\texttt{\$ echo ``aa bb cc'' | tp0 -s ab}\\
\indent\texttt{a b cc}\\

\indent\texttt{\$ echo ``aa bb cc'' | tp0 -s ab 12}\\
\indent\texttt{1 2 cc}\\

\indent\texttt{\$ echo ``Hola mundo'' |tp0 -d Ho}\\
\indent\texttt{la mund}\\

\indent\texttt{\$ echo ``aabbcc'' | tp0 -ds a b}\\
\indent\texttt{bcc}\\

\indent\texttt{\$ echo ``aaaa'' | tp0 -s a b}\\
\indent\texttt{b}\\

\newpage
\newpage

\subsubsection{Consideraciones}
Ademas de la funcionalidad descripta en 5.1, puede implementarse los siguientes items:

\begin{itemize}
\item Opci\'{o}n \indent\texttt{-c}: COmplemento de los caracteres en el \textit{string1}.
\item Conjunto de caracteres.
\end{itemize}

\subsubsection{Portabilidad}
Como es usual, es necesario que la implementaci\'{o}n desarrollada provea un grado m\'{i}nimo de portabilidad. Para satisfacer esto, el programa deber\'{a} funcionar al menos en  NetBSD/pmax (usando el simulador GXEmul) y la versi\'{o}n de Linux usada para correr el simulador.

\subsection{Informe}
El informe deber\'{a} incluir:

\begin{itemize}
\item Documentaci\'{o}n relevante al dise\~no e implementaci\'{o}n del programa.
\item Comando(s) para compilar el programa.
\item Corridas de prueba y sus correspondientes comentarios.
\item El c\'{o}digo fuente completo.
\item El c\'{o}digo MIPS32 generado por el compilador.
\item Este enunciado.
\end{itemize}

\section{Documentaci\'{o}n y dise\~no}
La realizaci\'{o}n  de este informe se hizo con la herramienta TEX / LATEX. Tanto el arhivo en formato pdf como el archivo .tex se encuentran en la carpeta doc del cd entregado.

Para el dise\~no se utiliz\'{o} un parseador, creado por nosotros, que sirve para analizar los argumentos que se le pasan al programa a trav\'{e}s de la l\'{i}nea de comandos.

Luego se realiza la funci\'{o}n indicada y el programa devuelve errores en el caso correspondiente a trav\'{e}s de \textit{stderr}.

\section{C\'{o}digos de retorno y mensajes de error}
Si la ejecuci\'{o}n fue exitosa el c\'{o}digo de retorno es 0 (cero), de lo contrario se informa por \textit{stderr} el error ocurrido y se retorna un -1 (menos uno).

\newpage
\newpage

\section{Casos de prueba}
Los casos de prueba utilizados son los que figuran a continuaci\'{o}n.\\

\indent\texttt{\$ echo ``3.14159192'' | ./tp0 1 9}\\
\indent\texttt{3.94959992}\\

\indent\texttt{\$ echo ``aba'' | ./tp0 aba 123}\\
\indent\texttt{323}\\

\indent\texttt{\$ echo ``aa bb cc'' | ./tp0 -s ab}\\
\indent\texttt{a b cc}\\

\indent\texttt{\$ echo ``aa bb cc'' | ./tp0 -s ab 12}\\
\indent\texttt{1 2 cc}\\

\indent\texttt{\$ echo ``Hola mundo'' | ./tp0 -d Ho}\\
\indent\texttt{la mund}\\

\indent\texttt{\$ echo ``aabbcc'' | ./tp0 -ds a b}\\
\indent\texttt{bcc}\\

\indent\texttt{\$ echo ``aaaa'' | ./tp0 -s a b}\\
\indent\texttt{b}\\

\indent\texttt{\$ echo ``bbb bbb'' | ./tp0 -s b}\\
\indent\texttt{b b}\\

\indent\texttt{\$echo ``bbbaaabb bbabbbbbbb'' | ./tp0 -ds b a}\\
\indent\texttt{a a}\\

\indent\texttt{\$echo ``f'' | ./tp0 aba def}\\
\indent\texttt{f}\\

\indent\texttt{\$echo ``abaco'' | ./tp0 -s a}\\
\indent\texttt{abaco}\\

\indent\texttt{\$echo ``abbbbbabbaaaco'' | ./tp0 -s a b}\\
\indent\texttt{bco}\\

\indent\texttt{\$echo ``bbbbbbaaaaaaabbb'' | ./tp0 -ds a b}\\
\indent\texttt{b}\

\indent\texttt{\$echo ``abaco baco abaco'' | ./tp0 a b}\\
\indent\texttt{bbbco bbco bbbco}\

\newpage
\newpage

\section{Instalaci\'{o}n}

La instalaci\'{o}n se encuentra automatizada por medio de un script de shell.
Para instalar el software se debe proceder de la siguiente manera:

\begin{itemize}
\item Descomprimir el archivo en la carpeta donde se desee realizar la instalaci\'{o}n.
\item En el shell cambiar el directorio de trabajo actual hacia el directorio de instalaci\'{o}n.
\item Ingresar la orden \$sh ./run.sh. Esto dispara el script que realiza la
compilaci\'{o}n e instalaci\'{o}n del software. El binario es instalado en la carpeta bin del cd entregado.
\end{itemize}

\newpage
\newpage

\section{C\'{o}digo fuente}
A contiuaci\'{o}n se detalla el c\'{o}digo fuente implementado en lenguaje C que se encuentra en carpeta code del cd entregado:
\lstset{basicstyle=\scriptsize, commentstyle=\textit, numbers=left, numberstyle=\footnotesize, tabsize=4, breaklines=true}\\
\lstset{language=C} 
\lstset{extendedchars=false} 
\lstinputlisting {../code/tp0.c}

\subsection{C\'{o}digo MIPS generado por el compilador GCC}
El siguiente es el c\'{o}digo assembly generado en el emulador \textit{GXemul}, corriendo una m\'{a}quina de arquitectura \textit{MIPS} con el sistema operativo \textit{NetBSD}\\ 
que se encuentra en la carpeta code del cd entregado:
\lstset{basicstyle=\scriptsize, commentstyle=\textit, numbers=left, numberstyle=\footnotesize, tabsize=4, breaklines=true}
\lstinputlisting{../code/tp0.s}

\newpage
\newpage

\section{Conclusiones}
El presente logr\'{o} ser una introducci\'{o}n a distintas herramientas que se utilizar\'{a}n en la materia.
Entre las actividades realizadas podemos resaltar:

\begin{itemize}
\item Compilar c\'{o}digo C utilizando gcc tanto en Linux como en NetBSD.
\item Obtener el assembly generado por el compilador para la arquitectura MIPS.
\item Utilizar el emulador GXemul para emular una maquina MIPS.
\item Utilizar un tunel SSH tanto para trabajar en la consola del guest como para transferir archivos del host al guest y viceversa.
\item Desarrollar el presente informe en LATEX.
\end{itemize}

\end{document}
